\chapter{Simulacija kvantnog računala}

\section{Postojeći simulatori kvantnog računala}

Danas postoje biblioteke i \emph{toolkits} za simulaciju kvantnih računala kao što su Qiskit, QuTiP, staq ili neki od brojnih drugih od kojih se veliki broj može pronaći na \citep{simulatori}.  Mnogi nude razne funkcionalnosti kao što su analiza tijeka izvođenja logičkog kruga ili razne načine vizualizacije stanja sustava, ali isto tako znaju imati zbunjujuću dokumentaciju i neintuitivan način korištenja.

Qiskit, s druge strane, ima vrhunsku dokumentaciju i uglavnom vrlo praktično sučelje. Čak dopušta programiranje i izvršavanje kvantnih logičkih krugova na pravim IBM-ovim kvantnim računalima za koje je često potrebno čekati u redu za korištenje. Također nudi razne vizualizacije kvantnog logičkog kruga i stanja sustava, a uz korištenje nekih drugih Python biblioteka kao što je matplotlib, moguće je dodatno vizualizirati rezultate izvođenja.

Samostalno izrađen simulator u sklopu ovoga rada ima sučelje slično Qiskitu, no napisan je u jeziku C++ te je dizajniran da što jednostavnije omogući demonstraciju nekih od prethodno opisanih pojava i algoritama u ovom radu.

\section{Izrada simulatora kvantnog računala}

\subsection{SQS}

\textbf{Simple Quantum Simulator} ili \textbf{SQS}  osmišljen je kao \emph{header-only} biblioteka u jeziku C++. Kao takav je neovisan o platformi uz kompromis duljeg vremena prevođenja. Za operacije s vektorima i matricama SQS koristi biblioteku Eigen \citep{eigen} koja se također sastoji od isključivo \emph{header} datoteka što dodatno produljuje vrijeme prevođenja, ali je ono i dalje prihvatljivo. Za korištenje biblioteke potrebno je u zaglavlje programa staviti \emph{include} datoteke \emph{sqs.h}.

\subsection{Struktura}

SQS se temelji na tri glavne komponente koje se koriste za konstrukciju kvantnih logičkih krugova. To su \textbf{QOperator}, \textbf{QComponent} i \textbf{QCircuit}.

\subsubsection{QOperator}
Razred QOperator predstavlja kvantni operator nad jednim ili više kvantnih bitova. Enkapsulira matricu operatora i sadrži informaciju o tipu operatora koji pomaže objektu tipa QComponent integrirati ga u logički krug. Korisnik ne mora brinuti o tipu operatora. Za instanciranje QOperatora korisniku su na raspolaganju konstante i funkcije za konstruiranje često korištenih kvantnih operatora. Konstante su redom: \textit{Eye}, \textit{Hadamard}, \textit{PauliX}, \textit{PauliY}, \textit{PauliZ}, \textit{CX}, \textit{CY}, \textit{CZ}, \textit{Toffoli} i \textit{SWAP}. Od funkcija to su:
\lstset{language=c++, tabsize=2}
\begin{lstlisting}
QOperator CU(size_t controls, QOperator unitary);
QOperator Phase(double phase);
QOperator QFT(unsigned int qubits);
QOperator QFTDagger(unsigned int qubits);
\end{lstlisting}
Funkcija \textit{CU} prima dva argumenta: koliko ima upravljačkih bitova te unitarni operator kojim upravljaju. \textit{Phase} prima fazu operatora, a \textit{QFT} i \textit{QFTDagger} primaju nad koliko kvantnih bitova djeluju. Sve navedene konstante i funkcije nalaze se u datoteci \emph{ops.hpp} biblioteke. Matrice konstanti se nalaze u datoteci \emph{eigenconsts.hpp}, ali su donekle sakrivene od korisnika koristeći namespace \emph{sqs::Private}.

Korisnik također može stvoriti vlastiti QOperator inicijalizirajući ga s Eigen matricom u konstruktoru ili koristeći linearnu kombinaciju postojećih i vlastitih QOperatora i Eigen matrica. QOperator ne brine o svojoj unitarnosti te tu odgovornost ostavlja korisniku.

\subsubsection{QComponent}

QComponent je odgovoran za prvi korak integracije kvantnih operatora u kvantni logički krug. Funkcionira na način da se u njega dodavaju objekti tipa QOperator ili QComponent uz navedene indekse kvantnih bitova nad kojim djeluju.
\begin{lstlisting}
void add(QOperator qop, std::vector<unsigned int> qubitPos);
void add(QOperator qop, unsigned int qPos);
void add(QOperator qop, unsigned int qPos1, unsigned int qPos2);
void add(QOperator qop, unsigned int qPos1, unsigned int qPos2,
	unsigned int qPos3);
void add(QComponent qcomp);
\end{lstlisting}
Ukoliko se radi o operatoru s upravljačkim bitovima, potrebno je njih prve navesti, neovisno kojim redoslijedom.

QComponent nastoji što više operatora staviti u paralelu što smanjuje količinu matrica kojih je kasnije potrebno izračunati u koracima simulacije. Od ostalih funkcionalnosti, QComponent nudi:
\begin{lstlisting}
void setIterations(unsigned int times);
void reverse();
std::vector<unsigned int> getQubitRange();
const MX& calculateMatrix();
\end{lstlisting}
\emph{setIterations} postavlja koliko puta se komponenta treba ponoviti u logičkom krugu što je korisno za neke algoritme poput Groverovog. \emph{reverse} preokreće komponentu, \emph{getQubitRange} računa koje kvantne bitove komponenta koristi, a \emph{calculateMatrix} računa matrični prikaz komponente. Zadnje dvije funkcije koristi QCircuit kako bi uspješno simulirao logički krug. \emph{MX} i ostali nestandardni tipovi koji se koriste su definirani u datoteci \emph{eigenconsts.hpp} i predstavljaju samo kraći zapis tipova Eigen matrica.

\subsubsection{QCircuit}

QCircuit predstavlja kvantni logički krug u kojeg se ugrađuju komponente koje nastoji što je više moguće paralelizirati. Pri stvaranju objekta tipa QCircuit, u konstruktoru je potrebno navesti broj kvantnih bitova. Funkcionalnosti koje QCircuit nudi su:
\begin{lstlisting}
void add(QComponent qcomp);
void add(QOperator qop, std::vector<unsigned int> qubitPos);
void add(QOperator qop, unsigned int qPos1);
void add(QOperator qop, unsigned int qPos1, unsigned int qPos2);
void add(QOperator qop, unsigned int qPos1, unsigned int qPos2,
	unsigned int qPos3);
void reverse();
void execute();
void resetQubits();
void clearCircuit();
VX getStateVector();
std:.vector<double> probabilityVector();
std::map<unsigned int, unsigned int> measure(unsigned int times);
void measureAndDisplay(unsigned int times);
\end{lstlisting}
\emph{execute} pokreće simulaciju kvantnog logičkog kruga, odnosno računa sve potrebne matrice te ih množi vektorom stanja koji je na početku inicijaliziran u $\ket{0}$. \emph{resetQubits} postavlja vektor stanja u početno stanje. \emph{clearCircuit} uklanja sve komponente iz logičkog kruga. \emph{getStateVector} vraća trenutni vektor stanja logičkog kruga, dok \emph{probabilityVector} vraća vektor vjerojatnosti. \emph{measure} vrši mjerenje onoliko puta koliko je navedeno argumentom te vraća rezultate mjerenja kao mapu. \emph{measureAndDisplay} radi isto što i \emph{measure}, samo što ne vraća mapu nego odmah prikazuje rezultate mjerenja na standardni izlaz.

\subsection{Izazovi pri implementaciji}

U suštini, simulator se može svesti na računanje tenzorskog produkta i matrično množenje, no situacija ipak nije toliko jednostavna. Sve matrične reprezentacije kvantnih operatora koji djeluju na dva ili više kvantnih bitova pretpostavljaju točno određen raspored bitova. Na primjer CNOT i Toffolijeva vrata pretpostavljaju da su svi bitovi jedan uz drugog i da su upravljački bitovi iznad ciljnog. To je vrlo ograničavajuće te je bilo potrebno pronaći način konstrukcije operatora koji djeluje na proizvoljno raspoređenim bitovima.

Za CU operator sa jednim upravljačkim bitom i jednim ciljnim bitom, to je uvijek moguće postići jer se on može rastaviti na zbroj tenzorskih produkata. Na primjer, CNOT operatori:


\begin{figure}[H]
\centering
\begin{quantikz}
\qw & \ctrl{1} & \qw & & & \targ{} & \qw \\
\qw & \targ{} & \qw & & &  \ctrl{-1} & \qw \\
\end{quantikz}
\caption{$CNOT_{1, 2}$ i $CNOT_{2,1}$ operatori}
\end{figure}

mogu se prikazati kao:
\[
CNOT_{1,2} = \ket{0}\bra{0}\otimes I_2 + \ket{1}\bra{1}\otimes \sigma_x \qquad
CNOT_{2,1} = I_2 \otimes\ket{0}\bra{0} +  \sigma_x \otimes\ket{1}\bra{1}
\]

Općenito, ako se između upravljačkog i ciljnog bita nalazi $n$ drugih operatora koji ne utječu na upravljački ili ciljni bit, zajednička matrica se može dobiti na način:
\[
U = \ket{0}\bra{0}\otimes U_1 \otimes \ldots \otimes U_n \otimes I_2 + \ket{1}\bra{1}\otimes U_1 \otimes \ldots \otimes U_n \otimes \sigma_x
\]
ili
\[
U = I_2 \otimes U_1 \otimes \ldots \otimes U_n \otimes \ket{0}\bra{0} +  \sigma_x \otimes U_1 \otimes \ldots \otimes U_n \otimes \ket{1}\bra{1}
\]
gdje $U_i$ mogu biti bilo koji unitarni operatori, a $I_2$ ako nema operatora.

Ovo naizgled rješava samo dio problema, ali je zapravo jedina stvar koja je potrebna za konstrukciju svih ostalih operatora koji djeluju na proizvoljno raspoređenim  bitovima.

Razlog tome je što se SWAP vrata mogu konstruirati od CNOT vrata. Vrijedi 
\begin{figure}[H]
\centering
\begin{quantikz}
\qw & \gate[swap]{} & \qw \\
\qw &  & \qw
\end{quantikz} =
\begin{quantikz}
\qw & \ctrl{1} & \targ{} & \ctrl{1} & \qw \\
\qw & \targ{} & \ctrl{-1} & \targ{} & \qw \\
\end{quantikz}
\caption{SWAP vrata prikazana pomoću CNOT vrata}
\end{figure}
Posljedica ove činjenice jest da je moguće zamijeniti bilo koja dva kvantna bita u logičkom krugu. Dakle, prije svake primjene višebitnog operatora, simulator napravi potrebne zamjene kako bi bitovi odgovarali ulazima operatora, bez da ikako mijenja matricu samog operatora. Takav način omogućuje korištenje logičkih vrata kao što su višeupravljačka Toffolijeva vrata:
\begin{figure}[H]
\centering
\begin{quantikz}
\qw & \ctrl{3} & \qw \\
\qw & \qw & \qw \\
\qw & \ctrl{1} & \qw \\
\qw & \targ{} & \qw \\
\qw & \ctrl{-1} & \qw
\end{quantikz}
\caption{Višeupravljačka Toffolijeva vrata}
\label{multitoffoli}
\end{figure}
Općenito matricu višeupravljačkog operatora je lagano konstruirati: svi elementi na dijagonali se postave u jedinicu, a u donji desni kut se postavi na matricu ciljnog operatora. Dakle, simulator bi vrata \ref{multitoffoli} konstruirao na način:
\begin{figure}[H]
\centering
\begin{quantikz}
\qw &\qw & \ctrl{3} & \qw & \qw \\
\qw & \swap{3} & \ctrl{2} & \swap{3} & \qw \\
\qw & \qw  & \ctrl{1} & \qw & \qw\\
\qw & \qw  & \targ{} & \qw & \qw \\
\qw & \targX{} & \qw & \targX{} & \qw
\end{quantikz}
\caption{Realizacija vrata sa slike \ref{multitoffoli} u simulatoru}
\end{figure}

\section{Primjeri simulacije kvantnih logičkih krugova}

\subsection{Superpozicija}

Radi lakšeg upoznavanja sa simulatorom, prvih par primjera će biti jednostavni. Simulator započinje svoj rad u stanju $\ket{0}$, stoga za postavljanje bitova u superpoziciju potrebno je primijeniti Hadamardov operator na sve bitove. Mjerenje sustava trebalo bi rezultirati približno jednakom raspodjelom svih vrijednosti. Logički krug koji treba simulirati:
\begin{figure}[H]
\centering
\begin{quantikz}
\lstick{$\ket{0}$} & \qw & \gate{H} & \qw & \meter{} \\
\lstick{$\ket{0}$} & \qw & \gate{H} & \qw & \meter{} \\
\lstick{$\ket{0}$} & \qw & \gate{H} & \qw & \meter{} \\
\lstick{$\ket{0}$} & \qw & \gate{H} & \qw & \meter{}
\end{quantikz}
\end{figure}

i njemu odgovarajući kod:
\begin{lstlisting}
#include "sqs/sqs.h"

using namespace sqs;

int main() {
	/* stvaranje kvantnog logickog kruga */
	QCircuit qc(4);

	/* dodavanje Hadamardovog operatora na sve bitove,
		moguce napraviti i u for petlji */
	qc.add(Hadamard, {0, 1, 2, 3});

	/* pokretanje simulatora i mjerenje stanja na kraju 1000 puta */
	qc.execute();
	qc.measureAndDisplay(1000);
	
	return 0;
}
\end{lstlisting}

Dobiveni ispis:
\begin{lstlisting}
|0000>: 63
|0001>: 56
|0010>: 64
|0011>: 62
|0100>: 61
|0101>: 68
|0110>: 59
|0111>: 68
|1000>: 55
|1001>: 72
|1010>: 50
|1011>: 82
|1100>: 59
|1101>: 63
|1110>: 62
|1111>: 56
\end{lstlisting}

\subsection{Spregnutost}

Spregnutost označava neseparabilno stanje kvantnih bitova te ga je iznenađujuće jednostavno dobiti; potrebna su samo dva operatora. Logički krug za dobivanje spregnutog sustava dva kvantna bita jest:
\begin{figure}[H]
\centering
\begin{quantikz}
\lstick{$\ket{0}$} & \qw & \gate{H} & \ctrl{1} & \meter{} \\
\lstick{$\ket{0}$} & \qw & \qw & \targ{} & \meter{}
\end{quantikz}
\end{figure}

\begin{lstlisting}
#include "sqs/sqs.h"

using namespace sqs;

int main() {

	QCircuit qc(2);

	/* dodavanje Hadamardovog operatora na qubit 0 */
	qc.add(Hadamard, 0);

	/* dodavanje CNOT operatora gdje je qubit 0 upravljacki,
		a 1 ciljni*/
	qc.add(CX, 0, 1);

	qc.execute();
	qc.measureAndDisplay(1000);
	
	return 0;
}
\end{lstlisting}
Dobiveni ispis:
\begin{lstlisting}
|00>: 498
|11>: 502
\end{lstlisting}

Za tri kvantna bita, primjer je sličan:
\begin{figure}[H]
\centering
\begin{quantikz}
\lstick{$\ket{0}$} & \gate{H} 	& \ctrl{1} 	& \qw 		& \meter{} \\
\lstick{$\ket{0}$} & \qw 			& \targ{} 	&  \ctrl{1}	& \meter{} \\
\lstick{$\ket{0}$} & \qw 			& \qw 		&  \targ{}	& \meter{}
\end{quantikz}
\end{figure}
uz odgovarajući kod
\begin{lstlisting}
QCircuit qc(3);

qc.add(Hadamard, 0);

qc.add(CX, 0, 1);
qc.add(CX, 1, 2);

qc.execute();
qc.measureAndDisplay(1000);
\end{lstlisting}
Dobiveni ispis:
\begin{lstlisting}
|000>: 487
|111>: 513
\end{lstlisting}
Nadalje, primjeri izvornog koda neće sadržavati elemente zaglavlja i funkcije main, nego samo relevantan dio koda.

\subsection{Deutschev algoritam}

Za Deutschev algoritam potrebno je implementirati funkciju oblika $f : \{0, 1\} \rightarrow \{0, 1\}$ te odrediti je li ona uravnotežena ili konstantna.

Za konstante funkcije, implementacija je takva da postaje jasno zašto se uvijek dobiva $\ket{0}$ kada se izmjeri prvi bit, odnosno prvih $n$ bitova u Deutsch-Josza generalizaciji algoritma. Za funkciju $f(x) = 0$ implementacija se sastoji od praznih žica, odnosno jediničnih matrica, dok $f(x) = 1$ ima $\sigma_x$ operator na izlaznom bitu. Očigledno je da izlazni bit uopće ne interagira sa ulazom što rezultira da ulaz na kraju evaluacije crne kutije ostaje nepromijenjen.
\begin{figure}[H]
\centering
\begin{minipage}{.5\textwidth}
\centering
\begin{quantikz}
\qw  & \qw & \qw \\
\qw & \qw & \qw
\end{quantikz}
\caption{$f(x) = 0$}
\end{minipage}%
\begin{minipage}{.5\textwidth}
\centering
\begin{quantikz}
\qw  & \qw & \qw \\
\qw & \gate{X} & \qw
\end{quantikz}
\caption{$f(x) = 1$}
\end{minipage}
\end{figure}

Implementacije uravnoteženih funkcija također nisu komplicirane:
\begin{figure}[H]
\centering
\begin{minipage}{.5\textwidth}
\centering
\begin{quantikz}
\qw & \ctrl{1} & \qw \\
\qw & \targ{} & \qw
\end{quantikz}
\caption{$f(x) = x$}
\end{minipage}%
\begin{minipage}{.5\textwidth}
\centering
\begin{quantikz}
\qw & \ctrl{1} & \qw & \qw \\
\qw & \targ{} & \gate{X} & \qw 
\end{quantikz}
\caption{$f(x) = \neg x$}
\end{minipage}
\end{figure}
Ovdje se vidi da će prevrtanje faze imati ključnu ulogu pošto je izlazni bit inicijaliziran u $\ket{-}$, a ulazni u $\ket{+}$. Primjenom CNOT operatora, izlazni bit će ulaznom bitu promijeniti fazu rezultirajući da će se ulazni bit pronaći u stanju $\ket{-}$ koje prolaskom kroz Hadamardov operator postaje stanjem $\ket{1}$.

Kvantni logički krug Deutschevog algoritma:
\begin{figure}[H]
\centering
\begin{quantikz}
\lstick{$\ket{0}$} & \qw & \gate{H} & \gate[wires=2][2cm]{U_f} \gateinput{$x$} \gateoutput{$x$} & \gate{H} & \meter{} \\
\lstick{$\ket{0}$} & \gate{X} & \gate{H} & \gateinput{$y$}\gateoutput{$y\oplus f(x)$} & \qw & \qw
\end{quantikz}
\end{figure}
gdje $U_f$ odgovara jednoj od spomenutih implementacija. Sve funkcije možemo testirati na način:
\begin{lstlisting}
#include "sqs/sqs.h"

using namespace sqs;

void Deutsch(int i) {
	QCircuit qc(2);

	qc.add(PauliX, 1);
	qc.add(Hadamard, 0, 1);

	switch(i) {
		case 0: // f(x) = 0
			break;
		case 1: // f(x) = 1
			qc.add(PauliX, 1);
			break;
		case 2: // f(x) = x
			qc.add(CX, 0, 1);
			break;
		case 3: // f(x) = ~x
			qc.add(CX, 0, 1);
			qc.add(PauliX, 1);
			break;
		default:
			return;
	}

	qc.add(Hadamard, 0);

	qc.execute();
	qc.measureAndDisplay(1000);
}

int main() {

	std::string funcs[] = {	"f(x) = 0", "f(x) = 1",
												"f(x) = x", "f(x) = ~x"};

	for(int i = 0; i < 4; ++i) {
		std::cout << funcs[i] << std::endl;
		Deutsch(i);
	}

	return 0;
}
\end{lstlisting}
Dobije se očekivani rezultat:
\begin{lstlisting}
f(x) = 0
|00>: 497
|10>: 503
f(x) = 1
|00>: 487
|10>: 513
f(x) = x
|01>: 508
|11>: 492
f(x) = ~x
|01>: 482
|11>: 518
\end{lstlisting}
Napomena: desni bit je rezultat algoritma; u skici kvantnog logičkog kruga, bit koji je najviši je u ispisu bit najmanje težine.

\subsection{Groverov algoritam}

\subsection{Kvantna estimacija faze}

\section{Prednosti i mane simulatora kvantnog računala}

Bilo koji simulator kvantnog računala nikada neće moći nadomjestiti pravo kvantno računalo što je posljedica same prirode takvog računala. Sustav od $n$ kvantnih bitova klasično računalo prikazuje vektorom dimenzije $2^n$, a svaki operator koji djeluje na taj vektor matricom dimenzije $2^n \times 2^n$. Uz problem pohrane takvih podataka, još je veći problem vrijeme izvršavanja operacija koje uključuju relativno velik broj kvantnih bitova za koje se nakon određene granice može reći da se neće nikada moći izvesti u razumnom vremenu na klasičnom računalu.

No, simulatori su daleko od toga da su beskorisni. Sama činjenica što su simulatori, a ne kvantni sustavi može pomoći pri analizi određenih svojstava i aspekata kvantnih sustava. Simulator je u svakom trenutku moguće zaustaviti i analizirati stanje sustava bez da narušimo to stanje. Također je bitno spomenuti da u simulatorima ne postoji\footnote{Osim ako to namjerno nije simulirano} kvantna dekoherencija koja smanjuje preciznost računanja u pravim kvantnim računalima.

U svakom slučaju, simulatori su odlični alati za eksperimentiranje, analizu i bolje shvaćanje kvantnih sustava, pogotovo ako pravo kvantno računalo nije lako dostupno.





























