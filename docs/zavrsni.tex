\documentclass[times, utf8, zavrsni, numeric]{fer}
\usepackage{booktabs}
\usepackage{float}
\usepackage{braket}
\usepackage{tikz}
\usepackage{quantikz}

%\ignorecitefornumbering
% za kod
\usepackage{listings}
\usepackage{xcolor}
\definecolor{codegreen}{rgb}{0,0.6,0}
\definecolor{codegray}{rgb}{0.5,0.5,0.5}
\definecolor{codepurple}{rgb}{0.58,0,0.82}
\definecolor{codeblue}{rgb}{0.23,0.22,0.97} % 60, 57, 247
\definecolor{backcolour}{rgb}{0.95,0.95,0.95}
\lstdefinestyle{mystyle}{
    backgroundcolor=\color{backcolour},   
    commentstyle=\color{codegreen},
    keywordstyle=\color{codeblue},
    numberstyle=\tiny\color{codegray},
    stringstyle=\color{codepurple},
    basicstyle=\ttfamily\footnotesize,
    breakatwhitespace=false,         
    breaklines=true,                 
    captionpos=b,                    
    keepspaces=true,                 
    numbers=left,                    
    numbersep=5pt,                  
    showspaces=false,                
    showstringspaces=false,
    showtabs=false,                  
    tabsize=2
}



\begin{document}

% TODO: Navedite broj rada.
\thesisnumber{228}

% TODO: Navedite naslov rada.
\title{Simulacija kvantnog računala}

% TODO: Navedite vaše ime i prezime.
\author{Dominik Matić}

%\maketitle

% Ispis stranice s napomenom o umetanju izvornika rada. Uklonite naredbu \izvornik ako želite izbaciti tu stranicu.
%\izvornik

% Dodavanje zahvale ili prazne stranice. Ako ne želite dodati zahvalu, naredbu ostavite radi prazne stranice.
\zahvala{}

\tableofcontents
 
%\chapter{Uvod}
%Uvod rada. Nakon uvoda dolaze poglavlja u kojima se obrađuje tema.

\chapter{Uvod}
\label{ch:uvod}
Neprestanim razvojem znanosti i tehnologije, kvantna računala postaju sve bitnijim dijelom računarstva. Predstavljaju sasvim drugačiji način računanja koji otvara vrata rješavanju mnogih problema koje klasično računalo, zbog same prirode problema, ili rješava znatno sporije ili uopće ne može riješiti u razumnom vremenu. Takvi problemi javljaju se u područjima kriptografije, umjetne inteligencije i strojnog učenja, računalne biologije, financija, simulacije kvantnih sustava, kao i u ostalim područjima računarstva kao što su algoritmi pretraživanja.

Područje kriptografije je posebno zanimljivo jer se većina današnjih sigurnosnih mehanizama interneta temelji na matematičkim problemima za koje se smatra da su teško izračunljivi. To su primarno problem faktorizacije velikih brojeva te problem diskretnog logaritma. Kvantno računalo efikasno rješava takve probleme, što je još devedesetih godina demonstrirao Peter Shor\citep{Shor:1994jg}. Naravno, iz toga slijedi da će kvantna računala biti velika prijetnja sigurnosti na internetu, no već su se počeli razvijati mehanizmi koji će biti otporni na napade kvantnim računalom[citati?] što  samo govori o tome koliko je kvantno računalo blizu da postane velikim dijelom računarstva.

Potencijal kvantnog računala poznat je desetljećima te su već smišljeni i detaljno opisani mnogi od algoritama prikladni za takav način računanja. Sve što je preostalo je izgraditi računalo koja će moći izvoditi te algoritme. Danas, IBM posjeduje kvantno računalo s najvećim brojem kvantnih bitova, njih čak 65, no ni to još nije dovoljno da bude korisno. Naime, problem nastaje s pojavom kvantne dekoherencije. Dekoherencija predstavlja gubitak informacije kvantog sustava zbog interakcije s okolinom što onemogućava precizno ili čak bilo kakvo računanje. No, kvantno računarstvo je trenutno veliki predmet istraživanja te se napretci ostvaruju skoro svaki dan. Mogućnost izgradnje kvantnog računala sa dovoljno velikim brojem kvantnih bitova je sve izglednija, kao primjer, IBM obećava do 2023. godine izgraditi kvantno računalo sa 1000 kvantnih bitova\citep{ibm:quantum}.

S ciljem demonstracije nekih od mogućnosti kvantnog računala, ovaj rad se u provm dijelu bavi osnovnim načelima kvantnog računala, odnosno matematičkim modelom kojim se opisuju kvantnomehaničke pojave koje omogućuju drugačiji način računanja. Nakon toga rad opisuje neke od najpoznatijih algoritama namijenjenih izvedbi na kvantnim računalima koji će zatim neki od njih biti implementirani i demonstrirani u samostalno izgrađenom simulatoru.

\chapter{Kvantno računalo}

\section{Polarizacija svjetlosti}

\section{Kvantni bit i Diracova notacija}



\chapter{Kvantni algoritmi}
\section{Groverov algoritam}
\section{Deutschov algoritam}
\section{Shorov algoritam}

\chapter{Simulacija kvantnog računala}

\section{Postojeći simulatori kvantnog računala}


\chapter{Zaključak}

Kvantna računala nude mogućnosti računanja koja su već desetljećima poznata, ali ih se tek sada počelo fizički ostvarivati. Posljedice nadolazećeg kvantnog doba imaju potencijala promijeniti svijet kao što se to dogodilo kod klasičnog računala, ali nisu tu zamijeniti ih, već ponuditi nešto sasvim novo. Područje kvantnog računarstva još je uvijek mlado i potrebno je uložiti puno truda u istraživanje i eksperimentiranje s novom tehnologijom i mogućnostima koje se otvaraju. Veliku pozornost treba posvetiti područjima umjetne inteligencije i strojnog učenja kojima se ovaj rad nije bavio, ali ih je bitno spomenuti.

Izgrađeni simulator u sklopu ovoga rada definitivno je moguće još unaprijediti i optimizirati; osmišljen je samo kao alat za demonstraciju nekih pojmova i algoritama i kao takav ima potencijala biti puno boljim te autor planira nastaviti raditi na njemu. S tim na umu, ovaj rad se bavio samo osnovama kvantnog računarstva te je za dublje shvaćanje potrebno još istraživanja.



\bibliography{literatura}
\bibliographystyle{fer}

\begin{sazetak}
Kvantna računala uvode novi način računanja koji znatno proširuje mogućnosti klasičnih računala i kao takva sadrže veliki potencijal koji se tek nedavno počeo ostvarivati. Ovaj rad se bavi osnovnim principima rada kvantnog računala kao i nekim bitnim konceptima koji se javljaju u kvantni logičkim krugovima kao što su spregnutost, kvantni paralelizam i prevrtanje faze. Nadalje, rad opisuje Deutschev, Groverov i Shorov algoritam te na kraju na samostalno izgrađenom simulatoru demonstrira neke od spomenutih pojava i algoritama.

\kljucnerijeci{simulator kvantnog računala, kvantno računalo, simulator, kvantni bit, qubit, kvantni algoritam. kvantni paralelizam, prevrtanje faze}
\end{sazetak}

% TODO: Navedite naslov na engleskom jeziku.
\engtitle{Quantum computer simulation}
\begin{abstract}
Quantum computers introduce a new way of computing that significantly expands the capabilities of classical computers and as such contain great potential that has only recently begun to be realized. This paper deals with the basic principles of quantum computing as well as some of the important concepts that occur in quantum logic circuits such as quantum entanglement, quantum parallelism, and phase kickback. Furthermore, the paper describes famous algorithms devised by D. Deutsch, L. K. Grover and P. W. Shor and finally demonstrates some of the mentioned phenomena and algorithms on a self-built simulator.

\keywords{quantum computer simulator, quantum computer, simulator, quantum bit, qubit, quantum algorithm, quantum parallelism, phase kickback}
\end{abstract}

\end{document}
